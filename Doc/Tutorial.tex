% This is the main LaTeX file which is used to produce the Biopython
% Tutorial documentation.
%
% If you just want to read the documentation, you can pick up ready-to-go
% copies in both pdf and html format from:
%
% http://biopython.org/DIST/docs/tutorial/Tutorial.html
% http://biopython.org/DIST/docs/tutorial/Tutorial.pdf
%
% If you want to typeset the documentation, you'll need a standard TeX/LaTeX
% distribution (I use teTeX, which works great for me on Unix platforms).
% Additionally, you need HeVeA (or at least hevea.sty), which can be
% found at:
%
% http://pauillac.inria.fr/~maranget/hevea/index.html
%
% You will also need the pictures included in the document, some of
% which are UMLish diagrams created by Dia
% (http://www.lysator.liu.se/~alla/dia/dia.html).
% These diagrams are available from Biopython git in the original dia
% format, which you can easily save as .png format using Dia itself.
% They are also checked in as the png files, so if you make
% modifications to the original dia files, the png files should also be
% changed.
%
% Once you're all set, you should be able to generate pdf by running:
%
% pdflatex Tutorial.tex  (to generate the first draft)
% pdflatex Tutorial.tex  (to get the cross references right)
% pdflatex Tutorial.tex  (to get the table of contents right)
%
% To generate the html, you'll need HeVeA installed. You should be
% able to just run:
%
% hevea -fix Tutorial.tex
%
% However, on older versions of hevea you may first need to remove the
% Tutorial.aux file generated by LaTeX, then run hevea twice to get
% the references right.
%
% If you want to typeset this and have problems, please report them
% at biopython-dev@biopython.org, and we'll try to get things resolved. We
% always love to have people interested in the documentation!

\documentclass{report}
\usepackage{url}
\usepackage{fullpage}
\usepackage{hevea}
\usepackage{graphicx}

% make everything have section numbers
\setcounter{secnumdepth}{4}

% Make links between references
\usepackage{hyperref}
\newif\ifpdf
\ifx\pdfoutput\undefined
  \pdffalse
\else
  \pdfoutput=1
  \pdftrue
\fi
\ifpdf
  \hypersetup{colorlinks=true, hyperindex=true, citecolor=red, urlcolor=blue}
\fi

\begin{document}

\begin{htmlonly}
\title{Biopython Tutorial and Cookbook}
\end{htmlonly}
\begin{latexonly}
\title{
%Hack to get the logo on the PDF front page:
\includegraphics[width=\textwidth]{images/biopython.jpg}\\
%Hack to get some white space using a blank line:
~\\
Biopython Tutorial and Cookbook}
\end{latexonly}

\author{Jeff Chang, Brad Chapman, Iddo Friedberg, Thomas Hamelryck, \\
Michiel de Hoon, Peter Cock, Tiago Antao, Eric Talevich, Bartek Wilczy\'{n}ski}
\date{Last Update -- 16 February 2014 (Biopython 1.63+)}

%Hack to get the logo at the start of the HTML front page:
%(hopefully this isn't going to be too wide for most people)
\begin{rawhtml}
<P ALIGN="center">
<IMG ALIGN="center" SRC="images/biopython.jpg" TITLE="Biopython Logo" ALT="[Biopython Logo]" width="1024" height="288" />
</p>
\end{rawhtml}

\maketitle
\tableofcontents

\chapter{Introduction}
\label{chapter:introduction}
\include{Tutorial/chapter1}

\chapter{Quick Start -- What can you do with Biopython?}
\label{chapter:quick-start}
\include{Tutorial/chapter2}

\chapter{Sequence objects}
\label{chapter:Bio.Seq}
\include{Tutorial/chapter3}

\chapter{Sequence annotation objects}
\label{chapter:SeqRecord}
\include{Tutorial/chapter4}

\chapter{Sequence Input/Output}
\label{chapter:Bio.SeqIO}
\include{Tutorial/chapter5}

\chapter{Multiple Sequence Alignment objects}
\label{chapter:Bio.AlignIO}
\include{Tutorial/chapter6}

\chapter{BLAST}
\label{chapter:blast}
\include{Tutorial/chapter7}

\chapter{BLAST and other sequence search tools (\textit{experimental code})}
\label{chapter:searchio}
\include{Tutorial/chapter8}

\chapter{Accessing NCBI's Entrez databases}
\label{chapter:entrez}
\include{Tutorial/chapter9}

\chapter{Swiss-Prot and ExPASy}
\label{chapter:swiss_prot}
\include{Tutorial/chapter10}

\chapter{Going 3D: The PDB module}
\include{Tutorial/chapter11}

\chapter{Bio.PopGen: Population genetics}
\include{Tutorial/chapter12}

\chapter{Phylogenetics with Bio.Phylo}
\label{sec:Phylo}
\include{Tutorial/chapter13}

\chapter{Sequence motif analysis using Bio.motifs}
\include{Tutorial/chapter14}

\chapter{Cluster analysis}
\include{Tutorial/chapter15}

\chapter{Supervised learning methods}
\include{Tutorial/chapter16}

\chapter{Graphics including GenomeDiagram}
\label{chapter:graphics}
\include{Tutorial/chapter17}

\chapter{Cookbook -- Cool things to do with it}
\label{chapter:cookbook}
\include{Tutorial/chapter18}

\chapter{The Biopython testing framework}
\label{sec:regr_test}
\include{Tutorial/chapter19}

\chapter{Advanced}
\label{chapter:advanced}
\include{Tutorial/chapter20}

\chapter{Where to go from here -- contributing to Biopython}
\include{Tutorial/chapter21}

\chapter{Appendix: Useful stuff about Python}
\label{sec:appendix}
\include{Tutorial/chapter22}
