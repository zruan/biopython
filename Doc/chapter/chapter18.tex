\chapter{Codon Alignment (\textit{experimental code})}
\label{chap:codonalignment}

WARNING: This chapter of the Tutorial describes an experimental module in
Biopython. It is being included in Biopython and documented here in the
tutorial in a pre-final state to allow a period of feedback and refinement
before we declare it stable. Until then the details will probably change,
and any scripts using the current Bio.SearchIO would need to be updated.
Please keep this in mind!

This chapter is about Codon Alignment, which is a special case of
nucleotide alignment in which the trinucleotides correspond directly to
amino acids in the translated protein product. Codon Alignment carries
information that can be used for a variety of evolutionary analysis.

This chapter has been divided into four parts to explain the codon
alignment support in Biopython. First, a general introduction about the
basic classes in \verb|Bio.CodonAlign| will be given. Then, a typical
procedure of how to obtain a codon alignment within Biopython is
discussed. Next, some simple applications of codon alignment, such as
dN/dS ratio estimation and neutrality test and so forth will be covered.
Finally, IO support of codon alignment will help user to conduct
analysis that cannot be done within Biopython.

\section{CodonSeq Class}

\verb|Bio.CodonAlign.CodonSeq| class is the base class in Codon
Alignment. It is similar to \verb|Bio.Seq| but with some extra
attributes. To obtain a \verb|CodonSeq| object, you just need
to give a \verb|str| object of nucleotide sequence whose length is a
multiple of 3 (This can be violated if you have \texttt{rf\_table}
argument). For example:

%doctest
\begin{verbatim}
>>> from Bio.CodonAlign import CodonSeq
>>> codon_seq = CodonSeq("AAATTTCCCGGG")
>>> codon_seq
CodonSeq('AAATTTCCCGGG', Gapped(CodonAlphabet(), '-'))
\end{verbatim}

An error will raise up if the input sequence is not a multiple of 3.

\begin{verbatim}
>>> codon_seq = CodonSeq("AAATTTCCCGG")
Traceback (most recent call last):
  File "<stdin>", line 1, in <module>
  File "/biopython/Bio/CodonAlign/CodonSeq.py", line 81, in __init__
    assert len(self) % 3 == 0, "Sequence length is not a triple number"
AssertionError: Sequence length is not a triple number
\end{verbatim}

By default, \verb|Bio.CodonAlign.default_codon_alphabet| will be
assigned to \verb|CodonSeq| object if you don't specify any Alphabet.
This \verb|default_codon_alphabet| is gapped universal genetic code,
which will work in most cases. However, if you are analyzing data from
mitochondria, for instance, and are in need of assigning an special
codon alphabet by yourself, \verb|Bio.CodonAlign.CodonAlphabet| also
provides you an easy solution. All you need is to pick up a
\verb|CodonTable| object that is correct for your data. For example:

%doctest
\begin{verbatim}
>>> from Bio.CodonAlign import CodonSeq
>>> from Bio.CodonAlign.CodonAlphabet import get_codon_alphabet
>>> from Bio.Data.CodonTable import generic_by_id
# vertebrate mitochondria alphabet
>>> codon_alphabet = get_codon_alphabet(generic_by_id[2], gap_char="-")
>>> codon_seq1 = CodonSeq("AAA---CCCGGG", alphabet=codon_alphabet)
>>> codon_seq1
CodonSeq('AAA---CCCGGG', CodonAlphabet(Vertebrate Mitochondrial))
\end{verbatim}

The slice of \verb|CodonSeq| is exactly the same with \verb|Seq| and
it will always return a \verb|Seq| object if you sliced a
\verb|CodonSeq|. For example:

%cont-doctest
\begin{verbatim}
>>> codon_seq1
CodonSeq('AAA---CCCGGG', CodonAlphabet(Vertebrate Mitochondrial))
>>> codon_seq1[:6]
Seq('AAA---', DNAAlphabet())
>>> codon_seq1[1:5]
Seq('AA--', DNAAlphabet())
\end{verbatim}

As you might imagine, \verb|CodonSeq| is able to be translated into
amino acid sequence based on the \verb|CodonAlphabet| within it. In
fact, \verb|CodonSeq| does more than this. \verb|CodonSeq| object
has a \texttt{rf\_table} attribute that dictates how the
\verb|CodonSeq| will be translated (\texttt{rf\_table} will indicate
the starting position of each codon in the sequence). This is useful if
your sequence is known to have frameshift events (genomic DNA) or
pseudogene that has insertion or deletion. You might notice that in the
previous example, you haven't specify the \texttt{rf\_table} when initiate
a \verb|CodonSeq| object. In fact, \verb|CodonSeq| object will
automatically assign a \texttt{rf\_table} to the \texttt{CodonSeq} if
you don't say anything about it.


In the example, we didn't assign \texttt{rf\_table} to
\texttt{codon\_seq1}. By default, \verb|CodonSeq| will automatically
generate a \texttt{rf\_table} to the coding sequence assuming no
frameshift events. In this case, it is \texttt{{[}0, 3, 6{]}}, which
means the first codon in the sequence starts at position 0, the second
codon in the sequence starts at position 3, and the third codon in the
sequence starts at position 6. In \texttt{codon\_seq2}, we only have 8
nucleotides in the sequence, but with \texttt{rf\_table} option
specified. In this case, the third codon starts at the 5th position of
the sequence rather than the 6th. And the \texttt{translate()} function
will use the \texttt{rf\_table} to get the translated amino acid
sequence.

Another thing to keep in mind is that \texttt{rf\_table} will only be
applied to ungapped nucleotide sequence. This makes \texttt{rf\_table}
to be interchangeable between \verb|CodonSeq| with the same sequence
but different gaps inserted. For example,

%cont-doctest
\begin{verbatim}
>>> codon_seq1 = CodonSeq("AAACCC---GGG")
>>> codon_seq1.rf_table
[0, 3, 6]
>>> codon_seq1.translate()
'KPG'
>>> codon_seq1.full_translate()
'KP-G'
\end{verbatim}

We can see that the \texttt{rf\_table} of \texttt{codon\_seq1} is still
\texttt{{[}0, 3, 6{]}}, even though we have gaps added. The
\verb|translate()| function will skip the gaps and return the ungapped
amino acid sequence. If gapped protein sequence is what you need,
\verb|full_translate()| comes to help.

It is also easy to convert \verb|Seq| object to \verb|CodonSeq|
object, but it is the user's responsibility to ensure all the necessary
information is correct for a \verb|CodonSeq| (mainly \texttt{rf\_table}).

\begin{verbatim}
>>> from Bio.Seq import Seq
>>> codon_seq = CodonSeq()
>>> seq = Seq('AAAAAA')
>>> codon_seq.from_seq(seq)
CodonSeq('AAAAAA', CodonAlphabet(Standard))
>>> seq = Seq('AAAAA')
>>> codon_seq.from_seq(seq)
Traceback (most recent call last):
  File "<stdin>", line 1, in <module>
  File "/biopython/Bio/CodonAlign/CodonSeq.py", line 264, in from_seq
    return cls(seq._data, alphabet=alphabet)
  File "/biopython/Bio/CodonAlign/CodonSeq.py", line 80, in __init__
    assert len(self) % 3 == 0, "Sequence length is not a triple number"
AssertionError: Sequence length is not a triple number
>>> codon_seq.from_seq(seq, rf_table=(0, 2))
CodonSeq('AAAAA', CodonAlphabet(Standard))
\end{verbatim}

\section{CodonAlignment Class}

The \verb|CodonAlignment| class is another new class in
\verb|Codon.Align|. It's aim is to store codon alignment data and
apply various analysis upon it. Similar to
\verb|MultipleSeqAlignment|, you can use numpy style slice to a
\verb|CodonAlignment|. However, once you sliced, the returned result
will always be a \verb|MultipleSeqAlignment| object.

%doctest
\begin{verbatim}
>>> from Bio.CodonAlign import default_codon_alphabet, CodonSeq, CodonAlignment
>>> from Bio.Alphabet import generic_dna
>>> from Bio.SeqRecord import SeqRecord
>>> from Bio.Alphabet import IUPAC, Gapped
>>> a = SeqRecord(CodonSeq("AAAACGTCG", alphabet=default_codon_alphabet), id="Alpha")
>>> b = SeqRecord(CodonSeq("AAA---TCG", alphabet=default_codon_alphabet), id="Beta")
>>> c = SeqRecord(CodonSeq("AAAAGGTGG", alphabet=default_codon_alphabet), id="Gamma")
>>> codon_aln = CodonAlignment([a, b, c])
>>> print codon_aln
CodonAlphabet(Standard) CodonAlignment with 3 rows and 9 columns (3 codons)
AAAACGTCG Alpha
AAA---TCG Beta
AAAAGGTGG Gamma
>>> codon_aln[0]
ID: Alpha
Name: <unknown name>
Description: <unknown description>
Number of features: 0
CodonSeq('AAAACGTCG', CodonAlphabet(Standard))
>>> print codon_aln[:, 3]
A-A
>>> print codon_aln[1:, 3:10]
NucleotideAlphabet() alignment with 2 rows and 6 columns
---TCG Beta
AGGTGG Gamma
\end{verbatim}

You can write out \verb|CodonAlignment| object just as what you do
with \verb|MultipleSeqAlignment|.

\begin{verbatim}
>>> from Bio import AlignIO
>>> AlignIO.write(codon_aln, 'example.aln', 'clustal')
\end{verbatim}

An alignment file called \texttt{example.aln} can then be found in your
current working directory. You can write \verb|CodonAlignment| out in
any MSA format that Biopython supports.

Currently, you are not able to read MSA data as a
\verb|CodonAlignment| object directly (because of dealing with
\texttt{rf\_table} issue for each sequence). However, you can read the
alignment data in as a \verb|MultipleSeqAlignment| object and convert
them into \verb|CodonAlignment| object using \verb|from_msa()|
class method. For example,

\begin{verbatim}
>>> aln = AlignIO.read('example.aln', 'clustal')
>>> codon_aln = CodonAlignment()
>>> print codon_aln.from_msa(aln)
CodonAlphabet(Standard) CodonAlignment with 3 rows and 9 columns (3 codons)
AAAACGTCG Alpha
AAA---TCG Beta
AAAAGGTGG Gamma
\end{verbatim}

Note, the \verb|from_msa()| method assume there is no frameshift
events occurs in your alignment. Its behavior is not guaranteed if your
sequence contain frameshift events!!

There is a couple of methods that can be applied to
\verb|CodonAlignment| class for evolutionary analysis. We will cover
them more in \ref{subsec:app}.

\section{Build a Codon Alignment}

Building a codon alignment is the first step of many evolutionary
anaysis. But how to do that? \verb|Bio.CodonAlign| provides you an
easy funciton --- \verb|build()| to achieve all. The data you need to
prepare in advance is a protein alignment and a set of DNA sequences
that can be translated into the protein sequences in the alignment.

\verb|CodonAlign.build()| method requires two mandatory arguments. The
first one should be a protein \verb|MultipleSeqAlignment| object and
the second one is a list of nucleotide \verb|SeqRecord| object. By
default, \verb|CodonAlign.build()| assumes the order of the alignment
and nucleotide sequences are in the same. For example:

%doctest
\begin{verbatim}
>>> from Bio import CodonAlign
>>> from Bio.Alphabet import IUPAC
>>> from Bio.Align import MultipleSeqAlignment
>>> from Bio.SeqRecord import SeqRecord
>>> from Bio.Seq import Seq
>>> nucl1 = SeqRecord(Seq('AAATTTCCCGGG', alphabet=IUPAC.IUPACUnambiguousDNA()), id='nucl1')
>>> nucl2 = SeqRecord(Seq('AAATTACCCGCG', alphabet=IUPAC.IUPACUnambiguousDNA()), id='nucl2')
>>> nucl3 = SeqRecord(Seq('ATATTACCCGGG', alphabet=IUPAC.IUPACUnambiguousDNA()), id='nucl3')
>>> prot1 = SeqRecord(nucl1.seq.translate(), id='prot1')
>>> prot2 = SeqRecord(nucl2.seq.translate(), id='prot2')
>>> prot3 = SeqRecord(nucl3.seq.translate(), id='prot3')
>>> aln = MultipleSeqAlignment([prot1, prot2, prot3])
>>> codon_aln = CodonAlign.build(aln, [nucl1, nucl2, nucl3])
>>> print codon_aln
CodonAlphabet(Standard) CodonAlignment with 3 rows and 12 columns (4 codons)
AAATTTCCCGGG nucl1
AAATTACCCGCG nucl2
ATATTACCCGGG nucl3
\end{verbatim}

In the above example, \verb|CodonAlign.build()| will try to match
\texttt{nucl1} with \texttt{prot1}, \texttt{nucl2} with \texttt{prot2}
and \texttt{nucl3} with \texttt{prot3}, i.e., assuming the order of
records in \texttt{aln} and \texttt{{[}nucl1, nucl2, nucl3{]}} is the
same.

\verb|CodonAlign.build()| method is also able to handle key match. In
this case, records with same id are paired. For example:

%cont-doctest
\begin{verbatim}
>>> nucl1 = SeqRecord(Seq('AAATTTCCCGGG', alphabet=IUPAC.IUPACUnambiguousDNA()), id='nucl1')
>>> nucl2 = SeqRecord(Seq('AAATTACCCGCG', alphabet=IUPAC.IUPACUnambiguousDNA()), id='nucl2')
>>> nucl3 = SeqRecord(Seq('ATATTACCCGGG', alphabet=IUPAC.IUPACUnambiguousDNA()), id='nucl3')
>>> prot1 = SeqRecord(nucl1.seq.translate(), id='prot1')
>>> prot2 = SeqRecord(nucl2.seq.translate(), id='prot2')
>>> prot3 = SeqRecord(nucl3.seq.translate(), id='prot3')
>>> aln = MultipleSeqAlignment([prot1, prot2, prot3])
>>> nucl = {'prot1': nucl1, 'prot2': nucl2, 'prot3': nucl3}
>>> codon_aln = CodonAlign.build(aln, nucl)
>>> print codon_aln
CodonAlphabet(Standard) CodonAlignment with 3 rows and 12 columns (4 codons)
AAATTTCCCGGG nucl1
AAATTACCCGCG nucl2
ATATTACCCGGG nucl3
\end{verbatim}

This option is handleful if you read nucleotide sequences using
\verb|SeqIO.index()| method, in which case the nucleotide dict with be
generated automatically.

Sometimes, you are neither not able to ensure the same order or the same
id. \verb|CodonAlign.build()| method provides you an manual approach to
tell the program nucleotide sequence and protein sequence correspondance
by generating a \texttt{corr\_dict}. \texttt{corr\_dict} should be a
dictionary that uses protein record id as key and nucleotide record id
as item. Let's look at an example:

%cont-doctest
\begin{verbatim}
>>> nucl1 = SeqRecord(Seq('AAATTTCCCGGG', alphabet=IUPAC.IUPACUnambiguousDNA()), id='nucl1')
>>> nucl2 = SeqRecord(Seq('AAATTACCCGCG', alphabet=IUPAC.IUPACUnambiguousDNA()), id='nucl2')
>>> nucl3 = SeqRecord(Seq('ATATTACCCGGG', alphabet=IUPAC.IUPACUnambiguousDNA()), id='nucl3')
>>> prot1 = SeqRecord(nucl1.seq.translate(), id='prot1')
>>> prot2 = SeqRecord(nucl2.seq.translate(), id='prot2')
>>> prot3 = SeqRecord(nucl3.seq.translate(), id='prot3')
>>> aln = MultipleSeqAlignment([prot1, prot2, prot3])
>>> corr_dict = {'prot1': 'nucl1', 'prot2': 'nucl2', 'prot3': 'nucl3'}
>>> codon_aln = CodonAlign.build(aln, [nucl3, nucl1, nucl2], corr_dict=corr_dict)
>>> print codon_aln
CodonAlphabet(Standard) CodonAlignment with 3 rows and 12 columns (4 codons)
AAATTTCCCGGG nucl1
AAATTACCCGCG nucl2
ATATTACCCGGG nucl3
\end{verbatim}

We can see, even though the second argument of \verb|CodonAlign.build()|
is not in the same order with \texttt{aln} in the above example, the
\texttt{corr\_dict} tells the program to pair protein records and
nucleotide records. And we are still able to obtain the correct
\verb|CodonAlignment| object.

The underlying algorithm of \verb|CodonAlign.build()| method is very
similar to \texttt{pal2nal} (a very famous perl script to build codon
alignment). \verb|CodonAlign.build()| will first translate protein
sequences into a long degenerate regular expression and tries to find a
match in its corresponding nucleotide sequence. When translation fails,
it divide protein sequence into several small anchors and tries to match
each anchor to the nucleotide sequence to figure out where the mismatch
and frameshift events lie. Other options available for
\verb|CodonAlign.build()| includes \texttt{anchor\_len} (default 10) and
\texttt{max\_score} (maximum tolerance of unexpected events, default
10). You may want to refer the Biopython build-in help to get more
information about these options.

Now let's look at a real example of building codon alignment. Here we
will use epidermal growth factor (EGFR) gene to demonstrate how to
obtain codon alignment. To reduce your effort, we have already collected
EGFR sequences for Homo sapiens, Bos taurus, Rattus norvegicus,
Sus scrofa and Drosophila melanogaster. You can download them from
\href{http://zruanweb.com/egfr.zip}{here}. Uncomressing the
\texttt{.zip}, you will see three files. \texttt{egfr\_nucl.fa} is
nucleotide sequences of EGFR and \texttt{egfr\_pro.aln} is EGFR protein
sequence alignment in \texttt{clustal} format. The \texttt{egfr\_id}
contains id correspondance between protein records and nucleotide
records. You can then try the following code (make sure the files are in
your current python working directory):

\begin{verbatim}
>>> from Bio import SeqIO, AlignIO
>>> nucl = SeqIO.parse('egfr_nucl.fa', 'fasta', alphabet=IUPAC.IUPACUnambiguousDNA())
>>> prot = AlignIO.read('egfr_pro.aln', 'clustal', alphabet=IUPAC.protein)
>>> id_corr = {i.split()[0]: i.split()[1] for i in open('egfr_id').readlines()}
>>> aln = CodonAlign.build(prot, nucl, corr_dict=id_corr, alphabet=CodonAlign.default_codon_alphabet)
/biopython/Bio/CodonAlign/__init__.py:568: UserWarning: gi|47522840|ref|NP_999172.1|(L 449) does not correspond to gi|47522839|ref|NM_214007.1|(ATG)
  % (pro.id, aa, aa_num, nucl.id, this_codon))
>>> print aln
CodonAlphabet(Standard) CodonAlignment with 6 rows and 4446 columns (1482 codons)
ATGATGATTATCAGCATGTGGATGAGCATATCGCGAGGATTGTGGGACAGCAGCTCC...GTG gi|24657088|ref|NM_057410.3|
---------------------ATGCTGCTGCGACGGCGCAACGGCCCCTGCCCCTTC...GTG gi|24657104|ref|NM_057411.3|
------------------------------ATGAAAAAGCACGAG------------...GCC gi|302179500|gb|HM749883.1|
------------------------------ATGCGACGCTCCTGGGCGGGCGGCGCC...GCA gi|47522839|ref|NM_214007.1|
------------------------------ATGCGACCCTCCGGGACGGCCGGGGCA...GCA gi|41327737|ref|NM_005228.3|
------------------------------ATGCGACCCTCAGGGACTGCGAGAACC...GCA gi|6478867|gb|M37394.2|RATEGFR
\end{verbatim}

We can see, while building the codon alignment a mismatch event is
found. And this is shown as a UserWarning.

\section{Codon Alignment Application}
\label{subsec:app}

The most important application of codon alignment is to estimate
nonsynonymous substitutions per site (dN) and synonymous substitutions
per site (dS). \verb|CodonAlign| currently support three counting
based methods (NG86, LWL85, YN00) and maximum likelihood method to
estimate dN and dS. The function to conduct dN, dS estimation is called
\verb|cal_dn_ds()|. When you obtained a codon alignment, it is quite
easy to calculate dN and dS. For example (assuming you have EGFR codon
alignmnet in the python working space):

\begin{verbatim}
>>> from Bio.CodonAlign.CodonSeq import cal_dn_ds
>>> print aln
CodonAlphabet(Standard) CodonAlignment with 6 rows and 4446 columns (1482 codons)
ATGATGATTATCAGCATGTGGATGAGCATATCGCGAGGATTGTGGGACAGCAGCTCC...GTG gi|24657088|ref|NM_057410.3|
---------------------ATGCTGCTGCGACGGCGCAACGGCCCCTGCCCCTTC...GTG gi|24657104|ref|NM_057411.3|
------------------------------ATGAAAAAGCACGAG------------...GCC gi|302179500|gb|HM749883.1|
------------------------------ATGCGACGCTCCTGGGCGGGCGGCGCC...GCA gi|47522839|ref|NM_214007.1|
------------------------------ATGCGACCCTCCGGGACGGCCGGGGCA...GCA gi|41327737|ref|NM_005228.3|
------------------------------ATGCGACCCTCAGGGACTGCGAGAACC...GCA gi|6478867|gb|M37394.2|RATEGFR
>>> dN, dS = cal_dn_ds(aln[0], aln[1], method='NG86')
>>> print dN, dS
0.0209078305058 0.0178371876389
>>> dN, dS = cal_dn_ds(aln[0], aln[1], method='LWL95')
>>> print dN, dS
0.0203061425453 0.0163935691992
>>> dN, dS = cal_dn_ds(aln[0], aln[1], method='YN00')
>>> print dN, dS
0.0198195580321 0.0221560648799
>>> dN, dS = cal_dn_ds(aln[0], aln[1], method='ML')
>>> print dN, dS
0.0193877676103 0.0217247139962
\end{verbatim}

If you are using maximum likelihood methdo to estimate dN and dS, you
are also able to specify equilibrium codon frequency to \texttt{cfreq}
argument. Available options include \texttt{F1x4}, \texttt{F3x4} and
\texttt{F61}.

It is also possible to get dN and dS matrix or a tree from a
\verb|CodonAlignment| object.

\begin{verbatim}
>>> dn_matrix, ds_matrix = aln.get_dn_ds_matrix()
>>> print dn_matrix
gi|24657088|ref|NM_057410.3|    0
gi|24657104|ref|NM_057411.3|    0.0209078305058 0
gi|302179500|gb|HM749883.1|     0.611523924924  0.61022032668   0
gi|47522839|ref|NM_214007.1|    0.614035083563  0.60401686212   0.0411803504059 0
gi|41327737|ref|NM_005228.3|    0.61415325314   0.60182631356   0.0670105144563 0.0614703609541 0
gi|6478867|gb|M37394.2|RATEGFR  0.61870883409   0.606868724887  0.0738690303483 0.0735789092792 0.0517984707257 0
gi|24657088|ref|NM_057410.3|    gi|24657104|ref|NM_057411.3|    gi|302179500|gb|HM749883.1| gi|47522839|ref|NM_214007.1|    gi|41327737|ref|NM_005228.3|    gi|6478867|gb|M37394.2|RATEGFR
>>> dn_tree, ds_tree = aln.get_dn_ds_tree()
>>> print dn_tree
Tree(rooted=True)
    Clade(branch_length=0, name='Inner5')
        Clade(branch_length=0.279185347322, name='Inner4')
            Clade(branch_length=0.00859186651689, name='Inner3')
                Clade(branch_length=0.0258992353629, name='gi|6478867|gb|M37394.2|RATEGFR')
                Clade(branch_length=0.0258992353629, name='gi|41327737|ref|NM_005228.3|')
            Clade(branch_length=0.0139009266768, name='Inner2')
                Clade(branch_length=0.020590175203, name='gi|47522839|ref|NM_214007.1|')
                Clade(branch_length=0.020590175203, name='gi|302179500|gb|HM749883.1|')
        Clade(branch_length=0.294630667432, name='Inner1')
            Clade(branch_length=0.0104539152529, name='gi|24657104|ref|NM_057411.3|')
            Clade(branch_length=0.0104539152529, name='gi|24657088|ref|NM_057410.3|')
\end{verbatim}

Another application of codon alignment that \verb|CodonAlign| supports
is Mcdonald-Kreitman test. This test compares the within species
synonymous substitutions and nonsynonymous substitutions and between
species synonymous substitutions and nonsynonymous substitutions to see
if they are from the same evolutionary process. The test requires gene
sequences sampled from different individuals of the same species. In the
following example, we will use Adh gene from fluit fly to demonstrate
how to conduct the test. The data includes 11 individuals from
D. melanogaster, 4 individuals from D. simulans and 12 individuals from
D. yakuba. The data is available from
\href{http://zruanweb.com/adh.zip}{here}. A function called
\verb|mktest()| will be used for the test.

%doctest examples lib:numpy
\begin{verbatim}
>>> from Bio import SeqIO, AlignIO
>>> from Bio.Alphabet import IUPAC
>>> from Bio.CodonAlign import build
>>> from Bio.CodonAlign.CodonAlignment import mktest

>>> pro_aln = AlignIO.read('adh.aln', 'clustal', alphabet=IUPAC.protein)
>>> p = SeqIO.index('drosophilla.fasta', 'fasta', alphabet=IUPAC.IUPACUnambiguousDNA())
>>> codon_aln = build(pro_aln, p)
>>> print codon_aln
CodonAlphabet(Standard) CodonAlignment with 27 rows and 768 columns (256 codons)
ATGGCGTTTACCTTGACCAACAAGAACGTGGTTTTCGTGGCCGGTCTGGGAGGCATT...ATC gi|9217|emb|X57365.1|
ATGGCGTTTACCTTGACCAACAAGAACGTGGTTTTCGTGGCCGGTCTGGGAGGCATT...ATC gi|9219|emb|X57366.1|
ATGGCGTTTACCTTGACCAACAAGAACGTGGTTTTCGTGGCCGGTCTGGGAGGCATT...ATC gi|9221|emb|X57367.1|
ATGGCGTTTACCTTGACCAACAAGAACGTGGTTTTCGTGGCCGGTCTGGGAGGCATT...ATC gi|9223|emb|X57368.1|
ATGGCGTTTACCTTGACCAACAAGAACGTGGTTTTCGTGGCCGGTCTGGGAGGCATT...ATC gi|9225|emb|X57369.1|
ATGGCGTTTACCTTGACCAACAAGAACGTGGTTTTCGTGGCCGGTCTGGGAGGCATT...ATC gi|9227|emb|X57370.1|
ATGGCGTTTACCTTGACCAACAAGAACGTGGTTTTCGTGGCCGGTCTGGGAGGCATT...ATC gi|9229|emb|X57371.1|
ATGGCGTTTACCTTGACCAACAAGAACGTGGTTTTCGTGGCCGGTCTGGGAGGCATT...ATC gi|9231|emb|X57372.1|
ATGGCGTTTACCTTGACCAACAAGAACGTGGTTTTCGTGGCCGGTCTGGGAGGCATT...ATC gi|9233|emb|X57373.1|
ATGGCGTTTACCTTGACCAACAAGAACGTGGTTTTCGTGGCCGGTCTGGGAGGCATT...ATC gi|9235|emb|X57374.1|
ATGGCGTTTACCTTGACCAACAAGAACGTGGTTTTCGTGGCCGGTCTGGGAGGCATT...ATC gi|9237|emb|X57375.1|
ATGGCGTTTACCTTGACCAACAAGAACGTGGTTTTCGTGGCCGGTCTGGGAGGCATT...ATC gi|9239|emb|X57376.1|
ATGGCGTTTACTTTGACCAACAAGAACGTGATTTTCGTTGCCGGTCTGGGAGGCATT...ATC gi|9097|emb|X57361.1|
ATGGCGTTTACTTTGACCAACAAGAACGTGATTTTCGTTGCCGGTCTGGGAGGCATT...ATC gi|9099|emb|X57362.1|
ATGGCGTTTACTTTGACCAACAAGAACGTGATTTTCGTTGCCGGTCTGGGAGGCATT...ATC gi|9101|emb|X57363.1|
ATGGCGTTTACTTTGACCAACAAGAACGTGATTTTCGTTGCCGGTCTGGGAGGCATC...ATC gi|9103|emb|X57364.1|
ATGTCGTTTACTTTGACCAACAAGAACGTGATTTTCGTGGCCGGTCTGGGAGGCATT...ATC gi|156879|gb|M17837.1|DROADHCK
ATGTCGTTTACTTTGACCAACAAGAACGTGATTTTCGTGGCCGGTCTGGGAGGCATT...ATC gi|156877|gb|M17836.1|DROADHCJ
ATGTCGTTTACTTTGACCAACAAGAACGTGATTTTCGTGGCCGGTCTGGGAGGCATT...ATC gi|156875|gb|M17835.1|DROADHCI
ATGTCGTTTACTTTGACCAACAAGAACGTGATTTTCGTGGCCGGTCTGGGAGGCATT...ATC gi|156873|gb|M17834.1|DROADHCH
ATGTCGTTTACTTTGACCAACAAGAACGTGATTTTCGTGGCCGGTCTGGGAGGCATT...ATC gi|156871|gb|M17833.1|DROADHCG
ATGTCGTTTACTTTGACCAACAAGAACGTGATTTTCGTTGCCGGTCTGGGAGGCATT...ATC gi|156863|gb|M19547.1|DROADHCC
ATGTCGTTTACTTTGACCAACAAGAACGTGATTTTCGTGGCCGGTCTGGGAGGCATT...ATC gi|156869|gb|M17832.1|DROADHCF
ATGTCGTTTACTTTGACCAACAAGAACGTGATTTTCGTGGCCGGTCTGGGAGGCATT...ATC gi|156867|gb|M17831.1|DROADHCE
ATGTCGTTTACTTTGACCAACAAGAACGTGATTTTCGTTGCCGGTCTGGGAGGCATT...ATC gi|156865|gb|M17830.1|DROADHCD
ATGTCGTTTACTTTGACCAACAAGAACGTGATTTTCGTTGCCGGTCTGGGAGGCATT...ATC gi|156861|gb|M17828.1|DROADHCB
ATGTCGTTTACTTTGACCAACAAGAACGTGATTTTCGTTGCCGGTCTGGGAGGCATT...ATC gi|156859|gb|M17827.1|DROADHCA

>>> print mktest([codon_aln[1:12], codon_aln[12:16], codon_aln[16:]])
0.00206457257254
\end{verbatim}

In the above example, \texttt{codon\_aln{[}1:12{]}} belongs to
D. melanogaster, \texttt{codon\_aln{[}12:16{]}} belongs to D. simulans
and \texttt{codon\_aln{[}16:{]}} belongs to D. yakuba. \verb|mktest()|
will return the p-value of the test. We can see in this case, 0.00206
\textless{}\textless{} 0.01, therefore, the gene is under strong
negative selection according to MK test.

\section{Future Development}

Because of the limited time frame for Google Summer of Code project,
some of the functions in \verb|CodonAlign| is not tested
comprehensively. In the following days, I will continue perfect the code
and several new features will be added. I am always welcome to hear your
suggestions and feature request. You are also highly encouraged to
contribute to the existing code. Please do not hesitable to email me
(zruan1991 at gmail dot com) when you have novel ideas that can make the
code better.
